\documentclass[journal,12pt,twocolumn]{IEEEtran}
\usepackage{setspace}
\usepackage{gensymb}
\usepackage{xcolor}
\usepackage{caption}
\singlespacing
\usepackage{siunitx}
\usepackage[cmex10]{amsmath}
\usepackage{mathtools}
\usepackage{hyperref}
\usepackage{amsthm}
\usepackage{mathrsfs}
\usepackage{txfonts}
\usepackage{stfloats}
\usepackage{cite}
\usepackage{cases}
\usepackage{subfig}
\usepackage{longtable}
\usepackage{multirow}
\usepackage{enumitem}
\usepackage{bm}
\usepackage{mathtools}
\usepackage{listings}
\usepackage{tikz}
\usetikzlibrary{shapes,arrows,positioning}
\usepackage{circuitikz}
\renewcommand{\vec}[1]{\boldsymbol{\mathbf{#1}}}
\DeclareMathOperator*{\Res}{Res}
\renewcommand\thesection{\arabic{section}}
\renewcommand\thesubsection{\thesection.\arabic{subsection}}
\renewcommand\thesubsubsection{\thesubsection.\arabic{subsubsection}}

\renewcommand\thesectiondis{\arabic{section}}
\renewcommand\thesubsectiondis{\thesectiondis.\arabic{subsection}}
\renewcommand\thesubsubsectiondis{\thesubsectiondis.\arabic{subsubsection}}
\hyphenation{op-tical net-works semi-conduc-tor}

\lstset{
language=Python,
frame=single, 
breaklines=true,
columns=fullflexible
}
\begin{document}
\theoremstyle{definition}
\newtheorem{theorem}{Theorem}[section]
\newtheorem{problem}{Problem}
\newtheorem{proposition}{Proposition}[section]
\newtheorem{lemma}{Lemma}[section]
\newtheorem{corollary}[theorem]{Corollary}
\newtheorem{example}{Example}[section]
\newtheorem{definition}{Definition}[section]
\newcommand{\BEQA}{\begin{eqnarray}}
        \newcommand{\EEQA}{\end{eqnarray}}
\newcommand{\define}{\stackrel{\triangle}{=}}
\newcommand{\myvec}[1]{\ensuremath{\begin{pmatrix}#1\end{pmatrix}}}
\newcommand{\mydet}[1]{\ensuremath{\begin{vmatrix}#1\end{vmatrix}}}
\bibliographystyle{IEEEtran}
\providecommand{\nCr}[2]{\,^{#1}C_{#2}} % nCr
\providecommand{\nPr}[2]{\,^{#1}P_{#2}} % nPr
\providecommand{\mbf}{\mathbf}
\providecommand{\pr}[1]{\ensuremath{\Pr\left(#1\right)}}
\providecommand{\qfunc}[1]{\ensuremath{Q\left(#1\right)}}
\providecommand{\sbrak}[1]{\ensuremath{{}\left[#1\right]}}
\providecommand{\lsbrak}[1]{\ensuremath{{}\left[#1\right.}}
\providecommand{\rsbrak}[1]{\ensuremath{{}\left.#1\right]}}
\providecommand{\brak}[1]{\ensuremath{\left(#1\right)}}
\providecommand{\lbrak}[1]{\ensuremath{\left(#1\right.}}
\providecommand{\rbrak}[1]{\ensuremath{\left.#1\right)}}
\providecommand{\cbrak}[1]{\ensuremath{\left\{#1\right\}}}
\providecommand{\lcbrak}[1]{\ensuremath{\left\{#1\right.}}
\providecommand{\rcbrak}[1]{\ensuremath{\left.#1\right\}}}
\theoremstyle{remark}
\newtheorem{rem}{Remark}
\newcommand{\sgn}{\mathop{\mathrm{sgn}}}
\newcommand{\rect}{\mathop{\mathrm{rect}}}
\newcommand{\sinc}{\mathop{\mathrm{sinc}}}
\providecommand{\abs}[1]{\left\vert#1\right\vert}
\providecommand{\res}[1]{\Res\displaylimits_{#1}}
\providecommand{\norm}[1]{\lVert#1\rVert}
\providecommand{\mtx}[1]{\mathbf{#1}}
\providecommand{\mean}[1]{E\left[ #1 \right]}
\providecommand{\fourier}{\overset{\mathcal{F}}{ \rightleftharpoons}}
\providecommand{\ztrans}{\overset{\mathcal{Z}}{ \rightleftharpoons}}
\providecommand{\system}[1]{\overset{\mathcal{#1}}{ \longleftrightarrow}}
\newcommand{\solution}{\noindent \textbf{Solution: }}
\providecommand{\dec}[2]{\ensuremath{\overset{#1}{\underset{#2}{\gtrless}}}}
\let\StandardTheFigure\thefigure
\def\putbox#1#2#3{\makebox[0in][l]{\makebox[#1][l]{}\raisebox{\baselineskip}[0in][0in]{\raisebox{#2}[0in][0in]{#3}}}}
\def\rightbox#1{\makebox[0in][r]{#1}}
\def\centbox#1{\makebox[0in]{#1}}
\def\topbox#1{\raisebox{-\baselineskip}[0in][0in]{#1}}
\def\midbox#1{\raisebox{-0.5\baselineskip}[0in][0in]{#1}}

\vspace{3cm}
\title{12.6.3.18}
\author{Lokesh Surana}
\maketitle
\section*{Class 12, Chapter 6, Exercise 3.18}

Q. A rectangular sheet of tin 45 cm by 24 cm is to be made into a box without top, by cutting off square from each corner and folding up the flaps. What should be the side of the square to be cut off so that the volume of the box is maximum ?

\solution The length of sheet is 45 cm $\brak{\text{Let } a}$ and breadth of sheet is 24 cm $\brak{\text{Let }b}$.

\begin{figure}[!htb]
    \centering
    \includegraphics[width=\columnwidth]{figs/rectangle1.png}
    \caption{Rectangular sheet}
    \label{fig:rectangular sheet}
\end{figure}

Here is the figure of how the rectangular sheet will look if we cut equal sized square from each corner.

\begin{figure}[!htb]
    \centering
    \includegraphics[width=\columnwidth]{figs/rectangle2.png}
    \caption{Rectangular sheet after cut}
    \label{fig:rectangular sheet cut}
\end{figure}

The new legnth, breadth and height of the box will be $a-2x$ $\brak{\text{Let }l}$, $b-2x$ $\brak{\text{Let }b}$ and $x$ $\brak{\text{Let }h}$ respectively.

\begin{align}
    \text{Volume of the box} &= l \times b \times h \\
    & = (a-2x) \times (b-2x) \times x \\
    \label{eq:volume} & = 4x^3 - 2\brak{a+b}x^2 + abx
\end{align}

From the figure , we can see that
\begin{align}
    \label{eq:x_range}
    0 < x \leq \min\brak{a,b}/2
\end{align}

Now, we have to find the side of the square to be cut off so that the volume of the box is maximum.
So we have to find maxima of equation \eqref{eq:volume}.

The derivative of the function is
\begin{align}
    \nabla f\brak{x} &= 12x^2 - 4\brak{a+b}x + ab 
\end{align}

For the given values of $a$ and $b$, the derivative of the function is
\begin{align}
    \nabla f\brak{x} &= 12x^2 - 4\brak{45+24}x + 45 \times 24\\ 
    &= 12x^2 - 276x + 1080\\
    &= 12\brak{x-18}{x-5}
\end{align}

We have to find maximum value of the function for $x$ in the range $0 < x \leq 12$, from \eqref{eq:x_range}.
Here is the plot of function looks line in the given range.

\begin{figure}[!htb]
    \centering
    \includegraphics[width=\columnwidth]{figs/plot.png}
    \caption{Plot of function}
    \label{fig:plot of function}
\end{figure}

Critical point of the function is $x=18$ and $x=5$.
Since $x=18$ is not in the given range, we can ignore it.

\begin{align}
    \nabla^2 f\brak{x} &= 24x - 276\\
    \nabla^2 f\brak{5} &= 24 \times 5 - 276\\ 
    &= -156 < 0
\end{align}

So, $x=5$ is the point of maixma of the function.

\end{document}