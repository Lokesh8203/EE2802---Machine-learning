\documentclass{beamer}
\def\inputGnumericTable{}
% Theme choice:
\usetheme{CambridgeUS}
\usepackage{amsmath}
\usepackage{mathtools}
\usepackage[latin1]{inputenc}
\usepackage{fullpage}
\usepackage{color}
\usepackage{array}
\usepackage{longtable}
\usepackage{calc}
\usepackage{multirow}
\usepackage{hhline}
\usepackage{ifthen}
\providecommand{\pr}[1]{\ensuremath{\Pr\left(#1\right)}}
\providecommand{\cdf}[2]{\ensuremath{\text{F}_{#1}\left(#2\right)}}
\providecommand{\erf}[1]{\ensuremath{\text{erf}(#1)}}
\setbeamertemplate{caption}[numbered]
% Title page details: 
\title[PT-100 ]{An Application of Machine Learning to model a Temperature Sensor(PT100)} 
\author{Lokesh Surana (ES20BTECH11017)}

\begin{document}

% Title page frame
\begin{frame}
    \titlepage
\end{frame}

% Outline frame
\begin{frame}{Outline}
    \tableofcontents
\end{frame}

\section{Introduction}
\begin{frame}{Aim}
    The modeling of the voltage-temperature characteristics of the PT-100 RTD (Resistance Temperature Detector) using least squares method.
\end{frame}

\section{Data}
\begin{frame}{Training data}
    \begin{table}[ht!]
        \input{tables/training_data.tex}
        \caption{Training data}
        \label{tab:Training data}
    \end{table}
\end{frame}

\begin{frame}{Validation data}
    \begin{table}[!ht]
        \centering
        \input{tables/validation_data.tex}
        \caption{Validation data}
        \label{tab:Validation data}
    \end{table}
\end{frame}

\end{document}