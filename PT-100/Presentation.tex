\documentclass{beamer}

\usepackage{setspace}
\usepackage{gensymb}
\singlespacing
\usepackage{amsmath}
\usepackage{bm}
\usepackage{cite}
\usepackage{cases}
\usepackage{subfig}
\usepackage{longtable}
\usepackage{multirow}
\usepackage{verbatim}
\usepackage{hyperref}
\usepackage{circuitikz}
\usepackage{listings}
\usepackage{color}    
\usepackage{array}    
\usepackage{longtable}
\usepackage{calc}     
\usepackage{multirow} 
\usepackage{hhline}   
\usepackage{ifthen}   
\DeclareMathOperator*{\Res}{Res}

% correct bad hyphenation here
\hyphenation{op-tical net-works semi-conduc-tor}
\def\inputGnumericTable{}                                 %%

\lstset{
%language=C,
frame=single, 
breaklines=true,
columns=fullflexible
}
%\lstset{
%language=tex,
%frame=single, 
%breaklines=true
%}
\hypersetup{
    colorlinks=true,
    linkcolor=black,
    urlcolor=blue,
}
\usetheme{CambridgeUS}

\DeclareMathOperator*{\argmax}{arg\,max}
\DeclareMathOperator*{\argmin}{arg\,min}
\newtheorem{proposition}{Proposition}[section]
\newcommand{\BEQA}{\begin{eqnarray}}
\newcommand{\EEQA}{\end{eqnarray}}
\newcommand{\define}{\stackrel{\triangle}{=}}
\bibliographystyle{IEEEtran}
\providecommand{\mbf}{\mathbf}
\providecommand{\pr}[1]{\ensuremath{\Pr\left(#1\right)}}
\providecommand{\qfunc}[1]{\ensuremath{Q\left(#1\right)}}
\providecommand{\sbrak}[1]{\ensuremath{{}\left[#1\right]}}
\providecommand{\lsbrak}[1]{\ensuremath{{}\left[#1\right.}}
\providecommand{\rsbrak}[1]{\ensuremath{{}\left.#1\right]}}
\providecommand{\brak}[1]{\ensuremath{\left(#1\right)}}
\providecommand{\lbrak}[1]{\ensuremath{\left(#1\right.}}
\providecommand{\rbrak}[1]{\ensuremath{\left.#1\right)}}
\providecommand{\cbrak}[1]{\ensuremath{\left\{#1\right\}}}
\providecommand{\lcbrak}[1]{\ensuremath{\left\{#1\right.}}
\providecommand{\rcbrak}[1]{\ensuremath{\left.#1\right\}}}
\theoremstyle{remark}
\newtheorem{rem}{Remark}
\newcommand{\sgn}{\mathop{\mathrm{sgn}}}
\providecommand{\abs}[1]{\left\vert#1\right\vert}
\providecommand{\res}[1]{\Res\displaylimits_{#1}} 
\providecommand{\norm}[1]{\left\lVert#1\right\rVert}
\providecommand{\mtx}[1]{\mathbf{#1}}
\providecommand{\mean}[1]{\mathbb{E}\left[ #1 \right]}   
\providecommand{\fourier}{\overset{\mathcal{F}}{ \rightleftharpoons}}
\providecommand{\system}[1]{\overset{\mathcal{#1}}{ \longleftrightarrow}}
\newcommand{\cosec}{\,\text{cosec}\,}
\providecommand{\dec}[2]{\ensuremath{\overset{#1}{\underset{#2}{\gtrless}}}}
\newcommand{\myvec}[1]{\ensuremath{\begin{pmatrix}#1\end{pmatrix}}}
\newcommand{\mydet}[1]{\ensuremath{\begin{vmatrix}#1\end{vmatrix}}}
\renewcommand{\vec}[1]{\mathbf{\boldsymbol{#1}}}
% Theme choice:
\newcounter{saveenumi}
\newcommand{\seti}{\setcounter{saveenumi}{\value{enumi}}}
\newcommand{\conti}{\setcounter{enumi}{\value{saveenumi}}}

% Title page details: 
\title[PT-100 ]{An Application of Machine Learning to model a Temperature Sensor(PT100)} 
\author{Lokesh Surana (ES20BTECH11017)}

\begin{document}

% Title page frame
\begin{frame}
    \titlepage
\end{frame}

% Outline frame
\begin{frame}{Outline}
    \tableofcontents
\end{frame}

\section{Introduction}
\begin{frame}{Aim}
    \begin{enumerate}
        \item The modeling of the voltage-temperature characteristics of the PT-100 RTD (Resistance Temperature Detector) using least squares method.
        \item In next slide we have training and validation data. This data have been recorded using voltage readings from serial monitor of arduino and temperature readings from a thermometer. 
    \end{enumerate}
\end{frame}

\section{Circuit}
\begin{frame}{Circuit Diagram}
    \begin{figure}[!ht]
        \centering
        \begin{circuitikz} \draw
            (0,0) to[battery1, l=$3.3\ V$, invert] (0,2)
            to[R, l^=$100\ \Omega$] (3,2) to[short, -o] (5,2);
            \draw (3,2) to[R, l^=$P\ \Omega$] (3,0)
            -- (0,0);
            \draw (3,0) to[short, -o] (5,0);
        \end{circuitikz}
        \caption{Schematic Circuit Diagram to Measure the Output of PT-100 ($P$).}
        \label{fig:ckt}
    \end{figure}
\end{frame}

%Explain why R =100 is being used
\begin{frame}{Why R = 100 $\Omega$?}
    \begin{enumerate}
        \item Source voltage V = 3.3Volt
        \item The most common type (PT100) has a resistance (P) of 100 $\Omega$ at 0$^\circ$ C and 138.4 $\Omega$ at 100 $^\circ$ C.
        \item So for any R we use voltage drop accross R will be, $\frac{VP}{P + R}$
        \item In terms of sensorvalue, $\frac{Sensorvalue \times P}{P + R} \times \frac{3.3}{1023}$
        \item So it we use very large R compared to P, the voltage drop will be very small even for signficant change in sensorvalue/Temperature.
        \item That's why we use R = 100 $\Omega$, a comparable value to P. 
    \end{enumerate} 
\end{frame}

\begin{frame}{Experiment}
    \begin{figure}[!ht]
         \centering
         \includegraphics[width=0.6\columnwidth]{figs/circuit.jpg}
     \end{figure}
     PT-100 circuit is connected to arduino 3.3V and ground pins. The 200$\Omega$ resistor is being used in circuit. The voltage/sensorvalue is read from A0 pin of arduino.
 \end{frame}

 \begin{frame}{Experiment}
    \begin{figure}[!ht]
         \centering
            \includegraphics[width=0.6\columnwidth]{figs/lcdcircuit.png}
     \end{figure}
        This is the LCD circuit which is used to display the temperature readings.
\end{frame}
\begin{frame}{Experiment}
    \begin{figure}[!ht]
         \centering
            \includegraphics[width=0.6\columnwidth]{figs/lcdsetup.jpg}
     \end{figure}
        This is the LCD circuit which is used to display the temperature readings.
\end{frame}

\section{Data}
\begin{frame}{Training data}
    \begin{table}[ht!]
        \input{tables/training_data.tex}
        \label{tab:Training data}
    \end{table}
\end{frame}

\begin{frame}{Validation data}
    \begin{table}[!ht]
        \centering
        \input{tables/validation_data.tex}
        \label{tab:Validation data}
    \end{table}
\end{frame}

\section{Model}
\begin{frame}{Model}
    The voltage reding for arduino varies as per temperature. The voltage reading can be modelled as
    \begin{align}
        V(T)       =  A + BT \\
        \text{this can be written in the form of } y  = \vec{x}^\top\vec{n}  \\
        y = V(T), \vec{n} = \myvec{A\\B}, \vec{x} = \myvec{1\\T}
    \end{align}
\end{frame}

\begin{frame}{Model}
    For multiple points,eqn (3) becomes
    \begin{align}
        \vec{X}^\top\vec{n} &= \vec{Y}                   \\
        \vec{X} & = \myvec{1           & 1 & \ldots & 1 \\T_1&T_2&\ldots&T_n} \\
        \vec{Y} & = \myvec{V\brak{T_1}                  \\V\brak{T_2}\\\vdots\\V\brak{T_n}}\\
        \vec{n} &= \myvec{A\\B}
    \end{align}
    Here $\vec{n}$ is the unknown, $\vec{X}$ and $\vec{Y}$ are readings.
\end{frame}

\begin{frame}{Model}
    We approximate $\vec{n}$ by using the least sqaures method. The Python code
    \texttt{codes/lsq.py} solves for $\vec{n}$.
    The calculated value of $\vec{n}$ is
    \begin{align}
        \vec{n} = \myvec{1.48 \\ 0.0029}
    \end{align}
    The linear model between temperature and voltage is given by
    \begin{align}
        V(T) = 1.48 + 0.0029T
    \end{align}
\end{frame}

\section{Data visulization}
\begin{frame}{Data visulization}
    \begin{figure}[!ht]
        \centering
        \includegraphics[width=0.6\columnwidth]{figs/train.png}
        \caption{Training data}
        \label{fig:Training data}
    \end{figure}
\end{frame}

\begin{frame}{Data visulization}
    \begin{figure}[!ht]
        \centering
        \includegraphics[width=0.6\columnwidth]{figs/test.png}
        \caption{Validation data}
        \label{fig:Validation data}
    \end{figure}

\end{frame}
\end{document}