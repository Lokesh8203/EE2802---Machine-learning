\documentclass[journal,12pt,twocolumn]{IEEEtran}

\usepackage{setspace}
\usepackage{gensymb}
\usepackage{xcolor}
\usepackage{caption}
\singlespacing
\usepackage{siunitx}
\usepackage[cmex10]{amsmath}
\usepackage{mathtools}
\usepackage{hyperref}
\usepackage{amsthm}
\usepackage{mathrsfs}
\usepackage{txfonts}
\usepackage{stfloats}
\usepackage{cite}
\usepackage{cases}
\usepackage{subfig}
\usepackage{longtable}
\usepackage{multirow}
\usepackage{enumitem}
\usepackage{mathtools}
\usepackage{listings}
\usepackage{tasks}
\usepackage{siunitx}
\sisetup{
   detect-family,
   detect-inline-family=math,
}
\usepackage{tikz}
\usetikzlibrary{shapes,arrows,positioning}
\usepackage{circuitikz}
\let\vec\mathbf
\DeclareMathOperator*{\Res}{Res}
\renewcommand\thesection{\arabic{section}}
\renewcommand\thesubsection{\thesection.\arabic{subsection}}
\renewcommand\thesubsubsection{\thesubsection.\arabic{subsubsection}}

\renewcommand\thesectiondis{\arabic{section}}
\renewcommand\thesubsectiondis{\thesectiondis.\arabic{subsection}}
\renewcommand\thesubsubsectiondis{\thesubsectiondis.\arabic{subsubsection}}
\hyphenation{op-tical net-works semi-conduc-tor}

\lstset{
language=Python,
frame=single, 
breaklines=true,
columns=fullflexible
}
\begin{document}
\theoremstyle{definition}
\newtheorem{theorem}{Theorem}[section]
\newtheorem{problem}{Problem}
\newtheorem{proposition}{Proposition}[section]
\newtheorem{lemma}{Lemma}[section]
\newtheorem{corollary}[theorem]{Corollary}
\newtheorem{example}{Example}[section]
\newtheorem{definition}{Definition}[section]
\newcommand{\BEQA}{\begin{eqnarray}}
\newcommand{\EEQA}{\end{eqnarray}}
\newcommand{\define}{\stackrel{\triangle}{=}}
\newcommand{\myvec}[1]{\ensuremath{\begin{pmatrix}#1\end{pmatrix}}}
\newcommand{\mydet}[1]{\ensuremath{\begin{vmatrix}#1\end{vmatrix}}}

\bibliographystyle{IEEEtran}
\providecommand{\nCr}[2]{\,^{#1}C_{#2}} % nCr
\providecommand{\nPr}[2]{\,^{#1}P_{#2}} % nPr
\providecommand{\mbf}{\mathbf}
\providecommand{\pr}[1]{\ensuremath{\Pr\left(#1\right)}}
\providecommand{\qfunc}[1]{\ensuremath{Q\left(#1\right)}}
\providecommand{\sbrak}[1]{\ensuremath{{}\left[#1\right]}}
\providecommand{\lsbrak}[1]{\ensuremath{{}\left[#1\right.}}
\providecommand{\rsbrak}[1]{\ensuremath{{}\left.#1\right]}}
\providecommand{\brak}[1]{\ensuremath{\left(#1\right)}}
\providecommand{\lbrak}[1]{\ensuremath{\left(#1\right.}}
\providecommand{\rbrak}[1]{\ensuremath{\left.#1\right)}}
\providecommand{\cbrak}[1]{\ensuremath{\left\{#1\right\}}}
\providecommand{\lcbrak}[1]{\ensuremath{\left\{#1\right.}}
\providecommand{\rcbrak}[1]{\ensuremath{\left.#1\right\}}}
\theoremstyle{remark}
\newtheorem{rem}{Remark}
\newcommand{\sgn}{\mathop{\mathrm{sgn}}}
\newcommand{\rect}{\mathop{\mathrm{rect}}}
\newcommand{\sinc}{\mathop{\mathrm{sinc}}}
\providecommand{\abs}[1]{\left\vert#1\right\vert}
\providecommand{\res}[1]{\Res\displaylimits_{#1}} 
\providecommand{\norm}[1]{\lVert#1\rVert}
\providecommand{\mtx}[1]{\mathbf{#1}}
\providecommand{\mean}[1]{E\left[ #1 \right]}
\providecommand{\fourier}{\overset{\mathcal{F}}{ \rightleftharpoons}}
\providecommand{\ztrans}{\overset{\mathcal{Z}}{ \rightleftharpoons}}
\providecommand{\system}[1]{\overset{\mathcal{#1}}{ \longleftrightarrow}}
\newcommand{\solution}{\noindent \textbf{Solution: }}
\providecommand{\dec}[2]{\ensuremath{\overset{#1}{\underset{#2}{\gtrless}}}}
\let\StandardTheFigure\thefigure
\def\putbox#1#2#3{\makebox[0in][l]{\makebox[#1][l]{}\raisebox{\baselineskip}[0in][0in]{\raisebox{#2}[0in][0in]{#3}}}}
     \def\rightbox#1{\makebox[0in][r]{#1}}
     \def\centbox#1{\makebox[0in]{#1}}
     \def\topbox#1{\raisebox{-\baselineskip}[0in][0in]{#1}}
     \def\midbox#1{\raisebox{-0.5\baselineskip}[0in][0in]{#1}}

\vspace{3cm}
\title{\LaTeX\ Assignment}
\author{Lokesh Surana}
\maketitle
\section*{Exercise 8.1}
\begin{enumerate}[itemsep=+2mm]
\item Find the area of the region bounded by the curve ${y}^2 = x$ and the line ${x = 1}$, ${x} = 4$ and the x-axis in the first quadrant.

\item Find the area of the region bounded by ${y}^2
= 9{x}$, ${x} = 2$, ${x} = 4$ and the x-axis in the
first quadrant.

\item Find the area of the region bounded by ${x}^2
= 4{y}$, ${y} = 2$, ${y} = 4$ and the y-axis in the
first quadrant.

\item Find the area of the region bounded by the ellipse \(\frac{{x}^2}{16}\ + \frac{{y}^2}{9} = 1\)

\item Find the area of the region bounded by the ellipse \(\frac{{x}^2}{4}\ + \frac{{y}^2}{9} = 1\)

\item Find the area of the region in the first quadrant enclosed by x-axis, line ${x} = \sqrt{3} y$ and the circle ${x}^2 + {y}^2
 = 4$.

\item Find the area of the smaller part of the circle ${x}^2 + {y}^2 = {a}^2$ cut off by the line ${x} = \frac{a}{\sqrt{2}}$.

\item The area between ${x} = {y}^2$ and ${x} = 4$ is divided into two equal parts by the line ${x} = {a}$, find the value of ${a}$.

\item Find the area of the region bounded by the parabola ${y} = {x}^2$ and ${y} = |{x}|$.

\item Find the area bounded by the curve ${x}^2 = 4{y}$ and the line ${x} = 4{y} - 2$

\item Find the area of the region bounded by the curve ${y}^2
= 4{x}$ and the line ${x} = 3$.

\end{enumerate}

Choose the correct answer in the following   Exercises 12 and 13.
\begin{enumerate} [resume]
\item Area lying in the first quadrant and bounded by the circle ${x}^2 + {y}^2 = 4$ and the lines ${x} = 0$ and ${x} = 2$ is \break \break
A) $\pi$  \hfill B) $\dfrac{\pi}{2}$    \break\break
C) $\dfrac{\pi}{3}$  \hfill D) $\dfrac{\pi}{4}$

\break
\item Area of the region bounded by the curve ${y}^2 = 4{x}$, y-axis and the line ${y} = 3$ is \break\break
A) $2$  \hfill B) $\dfrac{9}{4}$    \break\break
C) $\dfrac{9}{3}$  \hfill D) $\dfrac{9}{2}$

\end{enumerate}
\end{document}