\documentclass[journal,12pt,twocolumn]{IEEEtran}
\usepackage{setspace}
\usepackage{gensymb}
\usepackage{xcolor}
\usepackage{caption}
\singlespacing
\usepackage{siunitx}
\usepackage[cmex10]{amsmath}
\usepackage{mathtools}
\usepackage{hyperref}
\usepackage{amsthm}
\usepackage{mathrsfs}
\usepackage{txfonts}
\usepackage{stfloats}
\usepackage{cite}
\usepackage{cases}
\usepackage{subfig}
\usepackage{longtable}
\usepackage{multirow}
\usepackage{enumitem}
\usepackage{bm}
\usepackage{mathtools}
\usepackage{listings}
\usepackage{tikz}
\usetikzlibrary{shapes,arrows,positioning}
\usepackage{circuitikz}
\renewcommand{\vec}[1]{\boldsymbol{\mathbf{#1}}}
\DeclareMathOperator*{\Res}{Res}
\renewcommand\thesection{\arabic{section}}
\renewcommand\thesubsection{\thesection.\arabic{subsection}}
\renewcommand\thesubsubsection{\thesubsection.\arabic{subsubsection}}

\renewcommand\thesectiondis{\arabic{section}}
\renewcommand\thesubsectiondis{\thesectiondis.\arabic{subsection}}
\renewcommand\thesubsubsectiondis{\thesubsectiondis.\arabic{subsubsection}}
\hyphenation{op-tical net-works semi-conduc-tor}

\lstset{
language=Python,
frame=single, 
breaklines=true,
columns=fullflexible
}
\begin{document}
\theoremstyle{definition}
\newtheorem{theorem}{Theorem}[section]
\newtheorem{problem}{Problem}
\newtheorem{proposition}{Proposition}[section]
\newtheorem{lemma}{Lemma}[section]
\newtheorem{corollary}[theorem]{Corollary}
\newtheorem{example}{Example}[section]
\newtheorem{definition}{Definition}[section]
\newcommand{\BEQA}{\begin{eqnarray}}
        \newcommand{\EEQA}{\end{eqnarray}}
\newcommand{\define}{\stackrel{\triangle}{=}}
\newcommand{\myvec}[1]{\ensuremath{\begin{pmatrix}#1\end{pmatrix}}}
\newcommand{\mydet}[1]{\ensuremath{\begin{vmatrix}#1\end{vmatrix}}}
\bibliographystyle{IEEEtran}
\providecommand{\nCr}[2]{\,^{#1}C_{#2}} % nCr
\providecommand{\nPr}[2]{\,^{#1}P_{#2}} % nPr
\providecommand{\mbf}{\mathbf}
\providecommand{\pr}[1]{\ensuremath{\Pr\left(#1\right)}}
\providecommand{\qfunc}[1]{\ensuremath{Q\left(#1\right)}}
\providecommand{\sbrak}[1]{\ensuremath{{}\left[#1\right]}}
\providecommand{\lsbrak}[1]{\ensuremath{{}\left[#1\right.}}
\providecommand{\rsbrak}[1]{\ensuremath{{}\left.#1\right]}}
\providecommand{\brak}[1]{\ensuremath{\left(#1\right)}}
\providecommand{\lbrak}[1]{\ensuremath{\left(#1\right.}}
\providecommand{\rbrak}[1]{\ensuremath{\left.#1\right)}}
\providecommand{\cbrak}[1]{\ensuremath{\left\{#1\right\}}}
\providecommand{\lcbrak}[1]{\ensuremath{\left\{#1\right.}}
\providecommand{\rcbrak}[1]{\ensuremath{\left.#1\right\}}}
\theoremstyle{remark}
\newtheorem{rem}{Remark}
\newcommand{\sgn}{\mathop{\mathrm{sgn}}}
\newcommand{\rect}{\mathop{\mathrm{rect}}}
\newcommand{\sinc}{\mathop{\mathrm{sinc}}}
\providecommand{\abs}[1]{\left\vert#1\right\vert}
\providecommand{\res}[1]{\Res\displaylimits_{#1}}
\providecommand{\norm}[1]{\lVert#1\rVert}
\providecommand{\mtx}[1]{\mathbf{#1}}
\providecommand{\mean}[1]{E\left[ #1 \right]}
\providecommand{\fourier}{\overset{\mathcal{F}}{ \rightleftharpoons}}
\providecommand{\ztrans}{\overset{\mathcal{Z}}{ \rightleftharpoons}}
\providecommand{\system}[1]{\overset{\mathcal{#1}}{ \longleftrightarrow}}
\newcommand{\solution}{\noindent \textbf{Solution: }}
\providecommand{\dec}[2]{\ensuremath{\overset{#1}{\underset{#2}{\gtrless}}}}
\let\StandardTheFigure\thefigure
\def\putbox#1#2#3{\makebox[0in][l]{\makebox[#1][l]{}\raisebox{\baselineskip}[0in][0in]{\raisebox{#2}[0in][0in]{#3}}}}
\def\rightbox#1{\makebox[0in][r]{#1}}
\def\centbox#1{\makebox[0in]{#1}}
\def\topbox#1{\raisebox{-\baselineskip}[0in][0in]{#1}}
\def\midbox#1{\raisebox{-0.5\baselineskip}[0in][0in]{#1}}

\vspace{3cm}
\title{12.11.3.9}
\author{Lokesh Surana}
\maketitle
\section*{Class 12, Chapter 11, Exercise 3.9}

Q.9. Find the equation of the plane through the intersection of the planes $3{x} – {y} + 2{z} – 4 = 0 \text{ and } {x} + {y} + {z} – 2 = 0$ and the point $\myvec{2\\2\\1}$.

\solution The equation of given are given by
\begin{align}
	 {P}_1: \vec{n}_1^{\top}{\vec{x}} &= {c}_1\\
	 {P}_1: \vec{n}_1^{\top}{\vec{x}} &= {c}_2
\end{align}

The intersection of the planes is given by the solution of the system of equations
\begin{align}
    {P}: {P}_1 + \lambda {P}_2 &= 0 \\
    {P}: \vec{n}_1^{\top}{\vec{x}} - {c}_1 + \lambda\brak{\vec{n}_2^{\top}{\vec{x}} - {c}_2} &= 0
\end{align}

If this plane is passing through a point \vec{P}, then following will be satisfied

\begin{align}
	{P}: \brak{\vec{n}_1^{\top} + \lambda\vec{n}_2^{\top}}\vec{P} - \brak{{c}_1 + \lambda{c}_2} &= 0\\
    \implies \lambda = \frac{{c}_1 - {n}_1^{\top}\vec{P}}{{n}_2^{\top}\vec{P} - {c}_2} 
\end{align}

To get the value of $\lambda$, we need to substitute the value of $\vec{P}, \vec{n}_1, \vec{n}_2, {c}_1 \text{ and } {c}_2$ in the above equation.
The equation of given planes are given by
\begin{align}
    {P}_1: \myvec{3&-1&2}{\vec{x}} &= 4\\
    {P}_2: \myvec{1&1&1}{\vec{x}} &= 2
\end{align}

The intersection of the planes is given by the solution of the system of equations
\begin{align}
    {P}: {P}_1 + \lambda {P}_2 &= 0 \\
    {P}: \myvec{3+\lambda&-1+\lambda&2+\lambda}{\vec{x}} - {(4 + 2\lambda)} &= 0
\end{align}

These plane shall pass through point $\myvec{2\\2\\1}$, which means that
\begin{align}
    \lambda &= \frac{4 - \myvec{3&-1&2}{\myvec{2\\2\\1}}}{\myvec{1&1&1}{\myvec{2\\2\\1}} - 2} \\
    \lambda &= -\frac{2}{3}
\end{align}

The equation of plane is as follows:
\begin{align}
    \frac{1}{3}\myvec{7&-5&4}\vec{x} &= \frac{8}{3}\\
    \implies {P}: \myvec{7&-5&4}\vec{x} &= 8
\end{align}

\end{document}