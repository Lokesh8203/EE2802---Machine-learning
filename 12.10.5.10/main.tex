\documentclass[journal,12pt,twocolumn]{IEEEtran}
\usepackage{setspace}
\usepackage{gensymb}
\usepackage{xcolor}
\usepackage{caption}
\singlespacing
\usepackage{siunitx}
\usepackage[cmex10]{amsmath}
\usepackage{mathtools}
\usepackage{hyperref}
\usepackage{amsthm}
\usepackage{mathrsfs}
\usepackage{txfonts}
\usepackage{stfloats}
\usepackage{cite}
\usepackage{cases}
\usepackage{subfig}
\usepackage{longtable}
\usepackage{multirow}
\usepackage{enumitem}
\usepackage{bm}
\usepackage{mathtools}
\usepackage{listings}
\usepackage{tikz}
\usetikzlibrary{shapes,arrows,positioning}
\usepackage{circuitikz}
\renewcommand{\vec}[1]{\boldsymbol{\mathbf{#1}}}
\DeclareMathOperator*{\Res}{Res}
\renewcommand\thesection{\arabic{section}}
\renewcommand\thesubsection{\thesection.\arabic{subsection}}
\renewcommand\thesubsubsection{\thesubsection.\arabic{subsubsection}}

\renewcommand\thesectiondis{\arabic{section}}
\renewcommand\thesubsectiondis{\thesectiondis.\arabic{subsection}}
\renewcommand\thesubsubsectiondis{\thesubsectiondis.\arabic{subsubsection}}
\hyphenation{op-tical net-works semi-conduc-tor}

\lstset{
language=Python,
frame=single, 
breaklines=true,
columns=fullflexible
}
\begin{document}
\theoremstyle{definition}
\newtheorem{theorem}{Theorem}[section]
\newtheorem{problem}{Problem}
\newtheorem{proposition}{Proposition}[section]
\newtheorem{lemma}{Lemma}[section]
\newtheorem{corollary}[theorem]{Corollary}
\newtheorem{example}{Example}[section]
\newtheorem{definition}{Definition}[section]
\newcommand{\BEQA}{\begin{eqnarray}}
        \newcommand{\EEQA}{\end{eqnarray}}
\newcommand{\define}{\stackrel{\triangle}{=}}
\newcommand{\myvec}[1]{\ensuremath{\begin{pmatrix}#1\end{pmatrix}}}
\newcommand{\mydet}[1]{\ensuremath{\begin{vmatrix}#1\end{vmatrix}}}
\bibliographystyle{IEEEtran}
\providecommand{\nCr}[2]{\,^{#1}C_{#2}} % nCr
\providecommand{\nPr}[2]{\,^{#1}P_{#2}} % nPr
\providecommand{\mbf}{\mathbf}
\providecommand{\pr}[1]{\ensuremath{\Pr\left(#1\right)}}
\providecommand{\qfunc}[1]{\ensuremath{Q\left(#1\right)}}
\providecommand{\sbrak}[1]{\ensuremath{{}\left[#1\right]}}
\providecommand{\lsbrak}[1]{\ensuremath{{}\left[#1\right.}}
\providecommand{\rsbrak}[1]{\ensuremath{{}\left.#1\right]}}
\providecommand{\brak}[1]{\ensuremath{\left(#1\right)}}
\providecommand{\lbrak}[1]{\ensuremath{\left(#1\right.}}
\providecommand{\rbrak}[1]{\ensuremath{\left.#1\right)}}
\providecommand{\cbrak}[1]{\ensuremath{\left\{#1\right\}}}
\providecommand{\lcbrak}[1]{\ensuremath{\left\{#1\right.}}
\providecommand{\rcbrak}[1]{\ensuremath{\left.#1\right\}}}
\theoremstyle{remark}
\newtheorem{rem}{Remark}
\newcommand{\sgn}{\mathop{\mathrm{sgn}}}
\newcommand{\rect}{\mathop{\mathrm{rect}}}
\newcommand{\sinc}{\mathop{\mathrm{sinc}}}
\providecommand{\abs}[1]{\left\vert#1\right\vert}
\providecommand{\res}[1]{\Res\displaylimits_{#1}}
\providecommand{\norm}[1]{\lVert#1\rVert}
\providecommand{\mtx}[1]{\mathbf{#1}}
\providecommand{\mean}[1]{E\left[ #1 \right]}
\providecommand{\fourier}{\overset{\mathcal{F}}{ \rightleftharpoons}}
\providecommand{\ztrans}{\overset{\mathcal{Z}}{ \rightleftharpoons}}
\providecommand{\system}[1]{\overset{\mathcal{#1}}{ \longleftrightarrow}}
\newcommand{\solution}{\noindent \textbf{Solution: }}
\providecommand{\dec}[2]{\ensuremath{\overset{#1}{\underset{#2}{\gtrless}}}}
\let\StandardTheFigure\thefigure
\def\putbox#1#2#3{\makebox[0in][l]{\makebox[#1][l]{}\raisebox{\baselineskip}[0in][0in]{\raisebox{#2}[0in][0in]{#3}}}}
\def\rightbox#1{\makebox[0in][r]{#1}}
\def\centbox#1{\makebox[0in]{#1}}
\def\topbox#1{\raisebox{-\baselineskip}[0in][0in]{#1}}
\def\midbox#1{\raisebox{-0.5\baselineskip}[0in][0in]{#1}}

\vspace{3cm}
\title{12.10.5.10}
\author{Lokesh Surana}
\maketitle
\section*{Class 12, Chapter 10, Exercise 5.10}

Q.10. The two adjacent sides of a parallelogram are  ${2}\hat{i} -  {4}\hat{j} + {5}\hat{k}$ and ${1}\hat{i} -  {2}\hat{j} - {3}\hat{k}$.
Find the unit vector parallel to its diagonal. Also, find its area.

\solution The sides of the parallelogram are given as $\vec{a} = \myvec{2\\-4\\5}$ and $\vec{b} = \myvec{1\\-2\\-3}$.
\break
The diagonals of the parallelogram are given by
\begin{align}
    \vec{D}_1 = \vec{a} + \vec{b} = \myvec{3 \\-6\\2} \\
    \vec{D}_2 = \vec{a} - \vec{b} = \myvec{1 \\-2\\8}
\end{align}
The unit vectors parallel to the diagonals are given by
\begin{align}
    \hat{\vec{D}_1} = \frac{\vec{D}_1}{\norm{\vec{D}_1}} = \frac{\myvec{3 \\-6\\2}}{\sqrt{3^2 + 6^2 + 2^2}} = \myvec{\frac{3}{\sqrt{45}}\\-\frac{6}{\sqrt{45}}\\\frac{2}{\sqrt{45}}} \\
    \hat{\vec{D}_2} = \frac{\vec{D}_2}{\norm{\vec{D}_2}} = \frac{\myvec{1 \\-2\\8}}{\sqrt{1^2 + 2^2 + 8^2}} = \myvec{\frac{1}{\sqrt{69}}\\-\frac{2}{\sqrt{69}}\\\frac{8}{\sqrt{69}}}
\end{align}

The area of the parallelogram is given by the cross product or vector product of $\vec{A},\vec{B}$ is defined as
\begin{align}
    \vec{a} \times \vec{b} = \myvec{\mydet{\vec{a}_{23} & \vec{b}_{23} \\\vec{a}_{31}&\vec{b}_{31}\\\vec{a}_{12}&\vec{b}_{12}}}
\end{align}
Hence
\begin{align}
    \mydet{\vec{a}_{23} & \vec{b}_{23}} & =\mydet{-4 & -2 \\5&\-3}=22\\
    \mydet{\vec{a}_{31} & \vec{b}_{31}} & =\mydet{5  & -3 \\2&1}=-11\\
    \mydet{\vec{a}_{12} & \vec{b}_{12}} & =\mydet{2  & 1  \\-4&-2}=0
\end{align}
Substituting the values
\begin{align}
    \vec{a}\times\vec{b}=\myvec{22 \\-11\\0}
\end{align}
The area of the parallelogram is given by
\begin{align}
    \norm{\vec{a}\times\vec{b}} = \sqrt{22^2 + 11^2 + 0^2} = \sqrt{605}
\end{align}

\end{document}